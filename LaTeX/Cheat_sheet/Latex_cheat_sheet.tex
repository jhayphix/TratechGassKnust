\documentclass{beamer}
\usecolortheme{default}
\usetheme{Madrid}
\title{LaTeX Cheat Sheet}
\author{Tratech GASS KNUST}
\date{\today}

\begin{document}
	
	\begin{frame}
		\titlepage
	\end{frame}
	
	\begin{frame}{ Commands and Usage}
		\begin{itemize}
			\item \texttt{\textcolor{blue}{\textbackslash documentclass\{...\}}} - To specify the overall layout of a document.
			\item \texttt{\textcolor{blue}{\textbackslash title\{...\}}} - Sets the main title of the document.
			\item \texttt{\textcolor{blue}{\textbackslash author\{...\}}} - Sets the author of the document.
			\item \texttt{\textcolor{blue}{\textbackslash date\{\textbackslash today\}}} - Sets the date to the current date.
			\item \texttt{\textcolor{blue}{\textbackslash begin\{document\}}} - Initiates the document structure.
			\item \texttt{\textcolor{blue}{\textbackslash maketitle}} - To display the preamble i.e. title, author , date ...
			\item \texttt{\textcolor{blue}{\textbackslash tableofcontents}} - For table of contents
		\end{itemize}
		
	\end{frame}
	
	\begin{frame}{ Commands and Usage}
		\begin{itemize}
			\item \texttt{\textcolor{blue}{\%} - Indicate a comment}
			\item \texttt{\textcolor{blue}{\textbackslash par} -  It is used to create line breaks, similar to pressing "Enter" in a word processor}
			\item \texttt{\textcolor{blue}{\textbackslash noindent} - To suppress the indentation}
		\end{itemize}
	\end{frame}

	\begin{frame}{ Commands and Usage}
		\begin{itemize}
			\item \texttt{\textcolor{blue}{\textbackslash textbf\{...\}} - Used to make the enclosed text bold}
			\item \texttt{\textcolor{blue}{\textbackslash textit\{...\}} - Used to make the enclosed text italic}
			\item \texttt{\textcolor{blue}{\textbackslash underline\{...\}} -  Used to underline the enclosed text.}
		\end{itemize}
	\end{frame}

	\begin{frame}{ Commands and Usage}
		\begin{itemize}
			\item \texttt{\textcolor{blue}{\textbackslash section\{...\}}} - To define a new section in the document
			\item \texttt{\textcolor{blue}{\textbackslash subsection\{...\}}} - To define a new section under the main section
			\item \texttt{\textcolor{blue}{\textbackslash subsubsection\{...\}}} - To define a new section under the subsection
			\item \texttt{\textcolor{blue}{\textbackslash section *\{...\}}} - To create an unnumbered section or heading
		\end{itemize}
		
	\end{frame}
	
	\begin{frame}{Commands and Usage}
		
		\begin{itemize}
			\item \texttt{\textcolor{blue}{\textbackslash begin\{itemize\}}} - Starts a bulleted list environment.
			\item \texttt{\textcolor{blue}{\textbackslash begin\{enumerate\}}} - Initiates a numbered list environment.
			\item \texttt{\textcolor{blue}{\textbackslash begin\{equation\}}} - Begins a mathematical equation environment.
			\item \texttt{\textcolor{blue}{\textbackslash begin\{table\}}} - Initiates a table environment.
			\item \texttt{\textcolor{blue}{\textbackslash begin\{figure\}}} - Starts a figure or graphic environment.
		\end{itemize}
		
	\end{frame}

	\begin{frame}{Commands and Usage}
		
		\begin{itemize}
			\item \texttt{\textcolor{blue}{\$ ... \$}} - For inserting mathematical expressions within a line a text \\ [10pt]
			\item \texttt{\textcolor{blue}{\textbackslash[ ... \textbackslash]}} - For inserting mathematical expressions to appear as a centered equation on its own line
			\item \texttt{\textcolor{blue}{\textbackslash dots}} - For 3 dots (...)
			\item \texttt{\textcolor{blue}{\textbackslash sqrt\{...\}}} - For inserting square root
		\end{itemize}
		
	\end{frame}
	
	% Add more frames for additional commands...
	
\end{document}
