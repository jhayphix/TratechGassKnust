\documentclass{beamer}
\usetheme{Madrid}
\usecolortheme{default}


\title{Introduction to LaTeX}
\subtitle{Document Preparation System - Phase One}
\author{Tratech GASS KNUST}
\date{\today}

\begin{document}
	% Insert Title
	\maketitle
	\tableofcontents
	
	% --------------------------------------------------------------
	% Start of Introduction
	\section{Introduction}
	\begin{frame}{Overview}
		\begin{itemize}
			
			\item What is LaTeX? (pronounced “LAY-tek” or “LAH-tek”) \par \pause
			\small Latex is a powerful typesetting system for creating professional documents, including slides \pause \\ [10pt]
			
			\item What is Typesetting \par \pause
			Typesetting is the process of arranging and formatting text and images for printing or digital display, ensuring a visually appealing and readable layout. \pause
			
			\item Why use LaTeX?
			\begin{itemize}
				\item LaTeX excels in typesetting complex mathematics, tables, and technical content for the physical sciences.
				\item It provide efficient tool for document management
				\item Known for its ability to handle complex formatting and equations.
				\item LaTeX simplifies the production of complicated elements like table of contents, indexes, and lists of figures.
			\end{itemize}
			

		\end{itemize}
	\end{frame}	
	% End of Introduction
	
	% Over view
	\begin{frame}{Overview}
		\section*{When to Use Latex}
		
		\begin{itemize}
			\item Academic Documents
			\item Slides and Presentations
			\item Technical Writing
			\item Collaborative Projects
			\item Cross-Platform Sharing
		\end{itemize}
	\end{frame}
	
	
	% Getting Started
	\section{Getting Started}
	
	\begin{frame}{Installation}
		\begin{itemize}
			\item Download and install LaTeX distribution (e.g., \underline{MiKTeX}, TeX Live).
			\item Choose a LaTeX editor (e.g., \underline{TexStudio}, TeXShop, TeXworks, Overleaf)
		\end{itemize}
	\end{frame}
	% End of Getting Started
	
	% Basic Document Structure
	\section{Basic Document Structure}

	\begin{frame}{Document Class}
		\begin{itemize}
			\item \texttt{\textcolor{blue}{\textbackslash documentclass\{article\}}}: Common document class.
			\item Other classes: \texttt{report}, \texttt{book}, \texttt{beamer}, etc.
		\end{itemize}
	\end{frame}
	
	\begin{frame}{Special Symbols}
		Most symbols on the keyboard have their usual meaning. However the characters \par
		
		 \texttt{ \textbackslash \{\} \$ \^ \_ \% \& \# \~ } \par
		
		are used for special purposes within LATEX
	\end{frame}
	
	\begin{frame}{Document Structure}
		\begin{itemize}
			\item Preamble: \texttt{\textcolor{blue}{\textbackslash title}}, \texttt{\textcolor{blue}{\textbackslash author}}, \texttt{\textcolor{blue}{\textbackslash date}}.
			\item Body: \texttt{\textcolor{blue}{\textbackslash begin\{document\}} ... \textcolor{blue}{\textbackslash end\{document\}}}.
		\end{itemize}
	\end{frame}
	
	% End of Basic Document Structure
	
	% Section command
	\section{The Section Command}
	\begin{frame}{The \textbackslash section Command }
		The \texttt{\textcolor{blue}{\textbackslash section\{...\}}} command in LaTeX is used to define a new section in the document. \pause \par
		
		It creates a numbered or titled section heading, depending on the document class. \par
		
		Sections are typically hierarchical :
		\begin{itemize}
			\item \texttt{\textcolor{blue}{\textbackslash section\{...\}}}
			\item \texttt{\textcolor{blue}{\textbackslash subsection\{...\}}}
			\item \texttt{\textcolor{blue}{\textbackslash subsubsection\{...\}}}
			\item \texttt{\textcolor{blue}{\textbackslash section*\{...\}}}
		\end{itemize}
		
	\end{frame}
	% End of section command
	
	% Formatting Text
	\section{Formatting Text}
	
	\begin{frame}{Text Formatting}
		\begin{itemize}
			\item \texttt{\textcolor{blue}{\textbackslash textbf\{\}}}: Bold.
			\item \texttt{\textcolor{blue}{\textbackslash textit\{\}}}: Italics.
			\item \texttt{\textcolor{blue}{\textbackslash underline\{\}}}: Underline.
		\end{itemize}
	\end{frame}
	% End of Formatting Text
	
	% Lists
	\section{Lists}
	
	\begin{frame}{Lists}
		\begin{itemize}
			\item \texttt{\textcolor{blue}{\textbackslash begin\{itemize\}} ... \textcolor{blue}{\textbackslash end\{itemize\}}}: Bullet points.
			
			\item \texttt{\textcolor{blue}{\textbackslash begin\{enumerate\}} ... \textcolor{blue}{\textbackslash end\{enumerate\}}}: Numbered lists.
		\end{itemize}
	\end{frame}
	% End of Lists
	
	%Table of content
	\section{Adding table of content}
	
	\begin{frame}{Adding table of Content}
		\begin{itemize}
			\item \texttt{\textcolor{blue}{\textbackslash tableofcontents}} - For table of contents
		\end{itemize}
	\end{frame}
	% End of Mathematics
	
	% Mathematics
	 \section{Mathematics}
	
	 \begin{frame}{Math Mode - Adding math to latex}
		 \begin{itemize}
			 \item Inline math: \texttt{\textcolor{blue}{\$}...\textcolor{blue}{\$}}
			
			\item Display math: \texttt{\textcolor{blue}{\textbackslash[} ... \textcolor{blue}{\textbackslash]}}. 	
		 \end{itemize}
	 \end{frame}
	% End of Mathematics
	
	% Concluction
	\section{Conclusion}

	\begin{frame}{Conclusion}
		\begin{itemize}
			\item Recap of key LaTeX concepts.
			
			\item Additional resources for learning LaTeX.
		\end{itemize}
	\end{frame}
	% End of conclusion
	
\end{document}
